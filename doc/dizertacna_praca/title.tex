\begin{titlepage}
\phantom.

\bigskip

\begin{center}
{\sc\LARGE Žilinská Univerzita v Žiline}
\medskip

{\sc\Large Fakulta riadenia a informatiky}

\vfill\vfill\vfill\vfill

{\sc\LARGE Dizertačná práca}

\medskip

{\large Študijný odbor: {\bf Aplikovaná informatika}}
\end{center}


\vfill\vfill\vfill\vfill


\phantom.\hfill
\begin{minipage}{10cm}
\begin{center}
{\large\bf Ing. Michal Chovanec}

\medskip

{\large\bf
  Aproximácia funkcie ohodnotení v \\
  algoritmoch Q-learning \\
  neurónovou sieťou
}

\medskip

Vedúci: {\bf prof. Ing. Juraj Miček, PhD}

\medskip

%\hfill
%Reg.č. xxx/2008
\hfill
Apríl 2016
\hfill\phantom.
\end{center}
\end{minipage}
\hspace{1.7cm}\phantom.

\vspace{2.9cm}

\phantom.
\end{titlepage}


%--------------------------------------------------------------------------------------
%%% slovensky abstrakt

\begin{abstract}

\noindent
{\sc Michal Chovanec:} {\em   Aproximácia funkcie ohodnotení v algoritmoch Q-learning neurónovou sieťou}
[Dizertačná práca]

\noindent
Žilinská Univerzita v~Žiline,
Fakulta riadenia a informatiky,
Katedra technickej kybernetiky.

\noindent
Vedúci: prof. Ing. Juraj Miček, PhD

\noindent
FRI ŽU v~Žiline, 2016

\bigskip

Práca sa zaoberá aproximáciou funkcie ohodnotení konania agenta, v algoritmoch Q-learning.
V priestoroch s malým počtom stavov predstavuje vhodné riešenie tabuľka.
Pre prípady veľkého počtu stavov je tabuľkové riešenie ťažko vypočítateľné. Je tak nutné použiť
aproximáciu. Vhodným kandidátom je neurónová sieť. Tradičné riešenie doprednej
siete je však nepoužiteľné z dôvodov nemožnosti takúto sieť učiť. V práci
je preto venovaný priestor neurónovej sieti bázických funkcií ktorú už je možné
na daný problém trénovať iteračnými metódami.

\end{abstract}


%--------------------------------------------------------------------------------------
%%% anglicky abstrakt


\selectlanguage{english}
\begin{abstract}

\noindent
{\sc Michal Chovanec:} {\em Q-function aproximation in  Q-learning algortihms using neural network }

[Disertation thesis]

\noindent
University of Žilina,
Faculty of Management Science and Informatics,
Department of technical cybernetics.

\noindent
Tutor:  prof. Ing. Juraj Miček, PhD


\noindent
FRI ŽU v Žiline, 2016

\bigskip

This thesis is focused on Q-value function approximation, in Q-learning algorithms.
In state spaces with small numbers of states can be solution using table used well.
In cases with large numbers of states is table solution difficult to solve.
It is necessary to use approximation, where good candidate can be neural network.
Common solution, using feed forward neural network can't be used, because
impossible to learn this network. Thesis is focused to use basis functions neural
network, which can be learned using iterations methods.


\end{abstract}
\selectlanguage{slovak}


%%%%%%%%%%%%%%%%%%%%%%%%%%%%%%%%%%%%%%%%%%%%%%%%%%%%%%%%%%%%%%%%%%%%%%%
\newpage

\centerline{\bf Prehlásenie}

\vspace{2em}

\noindent
Prehlasujem, že som túto prácu napísal samostatne a že som uviedol
väčšinu použitých prameneňov a literatúry, z~ktorých som čerpal.

\vspace{2em}

\noindent
V~Žiline, dňa 22.4.2016
\hfill
Michal Chovanec
